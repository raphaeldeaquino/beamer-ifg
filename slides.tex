% Exemplo de slides formatado com LaTeX
\documentclass[qualificacaom]{formatacao/beamer-ifg}
% Opções da classe-ifg (ao usar mais de uma, separe por vírgulas): 
%   [tese]  	              	-> Tese de doutorado.
%   [dissertacao]     	-> Dissertação de mestrado (padrão).
%   [monografia]     	-> Monografia de especialização.
%   [tcc]   	              	-> Trabalho de conclusão de curso (graduação).
%   [qualificacaom]  	-> Qualificação de mestrado.
%   [qualificacaoe]   	-> Qualificação de especialização.
%   [qualificacaot]   	-> Qualificação de TCC.
%   [preprojetom]   	-> Pré-projeto de mestrado.
%   [preprojetoe]    	-> Pré-projeto de especialização.
%   [preprojetot]    	-> Pré-projeto de TCC.
%   [relatorio]    		-> Relatório de disciplina.

%INICIO DO DOCUMENTO ---------------------------------------------- %
\begin{document}

% EDITAR - Todos os cursos  ------------------------------------------- %
% -------- Autor(es)
\autor{Fulano de Tal da Silva} % (Nome completo. Ex.: José Feliciano da Silva)
\autorresumido{Fulano da Silva} % (Nome resumido. Ex.: José da Silva)
% Se houver segundo autor descomente e preencha o comando abaixo
%\sautor{Fulano de Tal} % (Ex.: José da Silva)
% Se houver terceiro autor descomente e preencha o comando abaixo
%\tautor{Fulano de Tal} % (Ex.: José da Silva)
% -------- Título e subtitulo
\titulo{Título do Trabalho}
% Bacharelado, Licenciatura, Curso Superior de Tecnologia, Mestrado Profissional
\tipocurso{Mestrado Profissional} 
\curso{Tecnologia, Gestão e Sustentabilidade}
% Inserir apenas no caso de relatório de disciplina
\disciplina{Seminários Interdisciplinar}
% -------- Local e data
\campus{Goiânia} % Câmpus em foi desenvolvido o trabalho
\dia{18} % Data da apresentação/defesa do trabalho
\mes{07} % Formato numérico: \dia{01}, \mes{01} e \ano{2007}
\ano{2020} % 
% -------- Orientador(a)
\orientador{Dr. Fulano de Tal}
% Unidade do(a) orientador(a) dentro da instituição. No caso de haver
% mais de um departamento inclua o departamento, a sigla da 
% instituição e o câmpus, como no modelo. Em câmpus com um único 
% departamento inclua somente a sigla da instituição e o câmpus.
\unidade{Departamento IV - IFG / Câmpus Goiânia} 
% Use o comandos a seguir se for Orientadora e não Orientador. 
% Neste caso comente o comando acima inserindo % no início
%\orientadora{\textless Nome da Orientadora\textgreater}
% -------- Co-orientador(a)
% Se não houver co-orientador comente os dois comandos abaixo
\coorientador{Dr. Fulano de Tal}
% Unidade do(a) orientador(a) dentro da instituição. No caso de haver
% mais de um departamento inclua o departamento, a sigla da 
% instituição e o câmpus, como no modelo. Em câmpus com um único 
% departamento inclua somente a sigla da instituição e o câmpus.
\unidadeco{Departamento IV - IFG / Câmpus Goiânia} 
% Use o comando a seguir se for Coorientadora e não Co-orientador. 
% Neste caso comente o primeiro comandos acima inserindo % no início 
%\coorientadora{\textless Nome da Coorientadora\textgreater}

% NÃO EDITAR. Se necessário edite apenas os arquivos na pasta "pre" %
\capa
\sumario

% EDITAR - Todos os cursos --------------------------------------- % 
% Arquivos com os slides. Para facilitar a organização é recomendado 
% criar um arquivo para cada seção. O nome do arquivo pode ser qualquer 
% um, desde que contenha apenas letras e números.
% Certifique-se de que os nomes coincidem com o que foi incluído 
\section{Exemplos de texto} % Seções são adicionadas para organizar sua apresentação em blocos discretos, todas as seções e subseções são automaticamente enviadas para o sumário como uma visão geral da apresentação, mas NÃO são exibidas na apresentação como slides separados

%------------------------------------------------

\subsection{Parágrafos e listas}

\begin{frame}
	\frametitle{Parágrafos de texto}
	
	Sed iaculis \alert{dapibus gravida}. Morbi sed tortor erat, nec interdum arcu. Sed id lorem lectus. Quisque viverra augue id sem ornare non aliquam nibh tristique. Aenean in ligula nisl. Nulla sed tellus ipsum. Donec vestibulum ligula non lorem vulputate fermentum accumsan neque mollis.
	
	\bigskip % Espaço em branco vertical
	
	% Quote example
	\begin{quote}
		Sed diam enim, sagittis nec condimentum sit amet, ullamcorper sit amet libero. Aliquam vel dui orci, a porta odio.\\
		--- Someone, somewhere\ldots
	\end{quote}
	
	\bigskip % Espaço em branco vertical
	
	Nullam id suscipit ipsum. Aenean lobortis commodo sem, ut commodo leo gravida vitae. Pellentesque vehicula ante iaculis arcu pretium rutrum eget sit amet purus. Integer ornare nulla quis neque ultrices lobortis.
\end{frame}

%------------------------------------------------

\begin{frame}
	\frametitle{Listas}
	\framesubtitle{Marcadores e listas numeradas} % Legenda opcional
	
	\begin{itemize}
		\item Lorem ipsum dolor sit amet, consectetur adipiscing elit
		\item Aliquam blandit faucibus nisi, sit amet dapibus enim tempus
		\begin{itemize}
			\item Lorem ipsum dolor sit amet, consectetur adipiscing elit
			\item Nam cursus est eget velit posuere pellentesque
		\end{itemize}
		\item Nulla commodo, erat quis gravida posuere, elit lacus lobortis est, quis porttitor odio mauris at libero
	\end{itemize}
	
	\bigskip % Espaço em branco vertical
	
	\begin{enumerate}
		\item Nam cursus est eget velit posuere pellentesque
		\item Vestibulum faucibus velit a augue condimentum quis convallis nulla gravida 
	\end{enumerate}
\end{frame}

%------------------------------------------------

\subsection{Blocos}

\begin{frame}
	\frametitle{Blocos de texto destacado}
	
	\begin{block}{Título do bloco}
		Lorem ipsum dolor sit amet, consectetur adipiscing elit. Integer lectus nisl, ultricies in feugiat rutrum, porttitor sit amet augue.
	\end{block}
	
	\begin{exampleblock}{Exemplo de título de bloco}
		Aliquam ut tortor mauris. Sed volutpat ante purus, quis accumsan.
	\end{exampleblock}
	
	\begin{alertblock}{Título do bloco de alerta}
		Pellentesque sed tellus purus. Class aptent taciti sociosqu ad litora torquent per conubia nostra, per inceptos himenaeos.
	\end{alertblock}
	
	\begin{block}{} % Bloco sem título
		Suspendisse tincidunt sagittis gravida. Curabitur condimentum, enim sed venenatis rutrum, ipsum neque consectetur orci.
	\end{block}
\end{frame}

%------------------------------------------------

\subsection{Colunas}

\begin{frame}
	\frametitle{Múltiplas Colunas}
	\framesubtitle{Subtítulo} % Subtítulo opcional
	
	\begin{columns}[c] % A opção "c" especifica o alinhamento vertical centralizado enquanto a opção "t" é usada para o alinhamento vertical superior
		\begin{column}{0.45\textwidth} % Largura da coluna esquerda
			\textbf{Cabeçalho}
			\begin{enumerate}
				\item Afirmação
				\item Explicação
				\item Exemplo
			\end{enumerate}
		\end{column}
		\begin{column}{0.5\textwidth} % Right column width
			Lorem ipsum dolor sit amet, consectetur adipiscing elit. Integer lectus nisl, ultricies in feugiat rutrum, porttitor sit amet augue. Aliquam ut tortor mauris. Sed volutpat ante purus, quis accumsan dolor.
		\end{column}
	\end{columns}
\end{frame}

%------------------------------------------------

\section{Tabelas e figuras}

\subsection{Tabela}

\begin{frame}
	\frametitle{Tabela}
	\framesubtitle{Subtítulo} % Subtítulo opcional
	
	\begin{table}
		\caption{Legenda da tabela}
		\begin{tabular}{l l l}
			\textbf{Tratamentos} & \textbf{Resposta 1} & \textbf{Resposta 2}\\
			Tratamento 1 & 0.0003262 & 0.562 \\
			Tratamento 2 & 0.0015681 & 0.910 \\
			Tratamento 3 & 0.0009271 & 0.296 
		\end{tabular}
		\fontetab{Elaborada pelo(a) autor(a)}
	\end{table}
\end{frame}

%------------------------------------------------

\subsection{Figure}

\begin{frame}
	\frametitle{Figure}
	
	\begin{figure}
		\caption{Logo do IFG Câmpus Goiânia.}
		\includegraphics[width=0.6\linewidth]{./fig/logogoiania}
		\fontefig{XXX}
	\end{figure}
\end{frame}
%------------------------------------------------

\section{Matemática}

\begin{frame}
	\frametitle{Definições \& Exemplos}
	
	\begin{definition}
		Um \alert{número primo} é um número que possui exatamente dois divisores.
	\end{definition}
	
	\smallskip 
	
	\begin{example}
		\begin{itemize}
			\item 2 é primo (dois divisores: 1 e 2).
			\item 3 é primo (dois divisores: 1 e 3).
			\item 4 não é primo (\alert{três} divisores: 1, 2 e 4).
		\end{itemize}
	\end{example}
	
	\smallskip 
	
	Você também pode usar os ambientes \texttt{teorema}, \texttt{lemma}, \texttt{proof} e \texttt{corollary}.
\end{frame}

%------------------------------------------------

\begin{frame}
	\frametitle{Teorema, Corolário e Prova}
	
	\begin{theorem}[Equivalência massa-energia]
		$E = mc^2$
	\end{theorem}
	
	\begin{corollary}
		$x + y = y + x$
	\end{corollary}
	
	\begin{proof}
		$\omega + \phi = \epsilon$
	\end{proof}
\end{frame}

%------------------------------------------------

\begin{frame}
	\frametitle{Equação}

	\begin{equation}
		\cos^3 \theta =\frac{1}{4}\cos\theta+\frac{3}{4}\cos 3\theta
	\end{equation}
\end{frame}

%------------------------------------------------

\begin{frame}[fragile] % NÉ necessário usar a opção fragile quando verbatim é usado no slide
	\frametitle{Verbatim}
	
	\begin{example}[Código de slide do teorema]
		\begin{verbatim}
			\begin{frame}
				\frametitle{Teorema}
				\begin{theorem}[Equivalência massa-energia]
					$E = mc^2$
				\end{theorem}
		\end{frame}\end{verbatim} % Deve estar na mesma linha
	\end{example}
\end{frame}

%------------------------------------------------

\begin{frame}
	Slide sem título.
\end{frame}
%------------------------------------------------

\section{Referências}

\begin{frame}
	\frametitle{Citando referências}
	
	Um exemplo do comando \texttt{\textbackslash cite} para citar na apresentação:
	
	\bigskip % Vertical whitespace
	
	Esta afirmação requer citação \cite{p1,p2}.
\end{frame}

\begin{frame} % Use [allowframebreaks] to allow automatic splitting across slides if the content is too long
	\frametitle{Referências}
	
	\begin{thebibliography}{99} % Beamer does not support BibTeX so references must be inserted manually as below, you may need to use multiple columns and/or reduce the font size further if you have many references
		\footnotesize % Reduce the font size in the bibliography
		
		\bibitem[Smith, 2022]{p1}
			John Smith (2022)
			\newblock Publication title
			\newblock \emph{Journal Name} 12(3), 45 -- 678.
			
		\bibitem[Kennedy, 2023]{p2}
			Annabelle Kennedy (2023)
			\newblock Publication title
			\newblock \emph{Journal Name} 12(3), 45 -- 678.
	\end{thebibliography}
\end{frame}



\end{document}

%------------------------------------------------------------------ %
%         F I M   D O  A R Q U I V O : slides.tex         %
%------------------------------------------------------------------ %