%------------------------------------------------

\section{Matemática}

\begin{frame}
	\frametitle{Definições \& Exemplos}
	
	\begin{definition}
		Um \alert{número primo} é um número que possui exatamente dois divisores.
	\end{definition}
	
	\smallskip 
	
	\begin{example}
		\begin{itemize}
			\item 2 é primo (dois divisores: 1 e 2).
			\item 3 é primo (dois divisores: 1 e 3).
			\item 4 não é primo (\alert{três} divisores: 1, 2 e 4).
		\end{itemize}
	\end{example}
	
	\smallskip 
	
	Você também pode usar os ambientes \texttt{teorema}, \texttt{lemma}, \texttt{proof} e \texttt{corollary}.
\end{frame}

%------------------------------------------------

\begin{frame}
	\frametitle{Teorema, Corolário e Prova}
	
	\begin{theorem}[Equivalência massa-energia]
		$E = mc^2$
	\end{theorem}
	
	\begin{corollary}
		$x + y = y + x$
	\end{corollary}
	
	\begin{proof}
		$\omega + \phi = \epsilon$
	\end{proof}
\end{frame}

%------------------------------------------------

\begin{frame}
	\frametitle{Equação}

	\begin{equation}
		\cos^3 \theta =\frac{1}{4}\cos\theta+\frac{3}{4}\cos 3\theta
	\end{equation}
\end{frame}

%------------------------------------------------

\begin{frame}[fragile] % NÉ necessário usar a opção fragile quando verbatim é usado no slide
	\frametitle{Verbatim}
	
	\begin{example}[Código de slide do teorema]
		\begin{verbatim}
			\begin{frame}
				\frametitle{Teorema}
				\begin{theorem}[Equivalência massa-energia]
					$E = mc^2$
				\end{theorem}
		\end{frame}\end{verbatim} % Deve estar na mesma linha
	\end{example}
\end{frame}

%------------------------------------------------

\begin{frame}
	Slide sem título.
\end{frame}